\documentclass{article}

\usepackage{amsmath,amsfonts,amsthm} % Math packages

\begin{document}
\title{Dynamics and Parameters of Arducopter}
\maketitle
\noindent
\section{Dynamics}
We define the position and velocity of the quadcopter in the intertial frame as $x = (x,y,z)^T$ and $\dot{x} = (\dot{x},\dot{y},\dot{z})^T$. Similarly, we define the roll, pitch, and yaw angles in the body frame as $\theta = (\phi,\theta,\psi)^T$, and the corresponding angular velocities equal to $\dot{\theta} = (\dot{\phi},\dot{\theta},\dot{\psi})^T$. We can convert these angular velocities into the angular velocity vector using the relation:
\begin{equation}
\omega = 	\begin{bmatrix}
			1 & 0 & -s_{\theta} \\
			0 & c_{\theta} & c_{\theta}s_{\phi} \\
			0 & -s_{\theta} & c_{\theta}c_{\phi}
			\end{bmatrix}
			\dot{\theta}
\end{equation}			
We can relate the body and inertial frame by a rotation matrix $R$ which goes from the body frame to the inertial frame. This matrix is derived by using the ZYZ Euler angle conventions and successively undoing the yaw, pitch and roll.
\begin{equation}
R =   	\begin{bmatrix} 
		c_{\theta}c_{\psi}-c_{\theta}s_{\phi}s_{\psi} & -c_{\psi}s_{\phi}-c_{\phi}c_{\theta}s_{\psi} & s_{\theta}s_{\psi} 	\\
		c_{\theta}c_{\psi}s_{\phi}+c_{\phi}s_{\psi} & c_{\phi}c_{\theta}c_{\psi}-s_{\phi}s_{\psi} & -c_{\psi}s_{\theta} 	\\
		s_{\phi}s_{\theta} & c_{\phi}s_{\theta} & c_{\theta}
		\end{bmatrix}\\		
\end{equation}
For a given vector $\vec{v}$ in the body frame, the corresponding vector is given by $R\vec{v}$ in the inertial frame.\\\\
Let $\omega_i$ represent the angular velocity of the $i$th propeller.  The thrust generated by each of the propellers (in the body frame) can be written as:
\begin{equation}
\label{thrust}
T_B = \sum\limits_{i=1}^4 T_i = k \begin{bmatrix} 0 \\ 0 \\ \sum\nolimits {\omega_i}^2 \end{bmatrix}
\end{equation}
where $k$ is some appropriately dimensioned constant. 
\subsection{Linear motion}
The global drag force on the quadcopter can be modeled as a force proportional to the linear velocity in each direction:
\begin{equation}
\label{friction}
F_D = \begin{bmatrix} -k_d\dot{x} \\ -k_d\dot{y} \\ -k_d\dot{z} \end{bmatrix}
\end{equation}
In the inertial frame, the acceleration of the quadcopter is due to thrust, gravity and linear friction. Thus, the equations of motion of the quadcopter for linear motion are given as:
\begin{equation}
\label{linear}
m\ddot{\vec{x}} = \begin{bmatrix}0 \\ 0 \\ -mg \end{bmatrix} + RT_B + F_D
\end{equation}
where $\vec{x}$ is the position of the quadcopter, $g$ is the acceleration due to gravity, $F_D$ is the drag force, and $T_B$ is the thrust vector in the body frame. The equations can be simplified as:
\begin{equation}
\begin{bmatrix} \ddot{x} \\ \ddot{y} \\ \ddot{z} \end{bmatrix} = 
\frac{1}{m} \begin{bmatrix} s_{\psi}s_{\theta} k\sum\nolimits \omega_i^2 - k_d\dot{x} \\\\
							-c_{\psi}s_{\theta} k\sum\nolimits \omega_i^2 - k_d\dot{y} \\\\
							c_{\theta} k\sum\nolimits \omega_i^2 - k_d\dot{z}-mg 
			\end{bmatrix}
\end{equation}
\subsection{Rotational Motion}
Let the propellers $i=1$ and $i=3$ be on the roll axis, spinning clockwise, and $i=2$ and $i=4$ be on the pitch axis, spinning anticlockwise. Then the contribution of torque on the quadcopter from the $i$th propeller, about the Z axis (yaw axis) in steady state flight ($\dot{\omega}\approx0$) is given by:
\begin{equation}
\tau_z = (-1)^{i+1}bw_i^2
\end{equation}
where $b$ is some appropriately dimensioned constant. The torque on the quadcopter in the body frame is then given by:
\begin{equation}
\label{torque}
\tau_B 	= \begin{bmatrix} \tau_{\phi} \\ \tau_{\theta} \\ \tau_{\psi} \end{bmatrix}
		= \begin{bmatrix} Lk(\omega_1^2 - \omega_3^2) \\ Lk(\omega_2^2-\omega_4^2) \\ b(\omega_1^2-\omega_2^2+\omega_3^2-				\omega_4^2) \end{bmatrix}
\end{equation}
where $L$ is the distance from the center of the quadcopter to any of the propellers. \\\\
The rotational equations of motion can be derived using Euler's equations for rigid body dynamics. Expressed in vector form, Euler's equations are written as:
\begin{equation}
I\dot{\omega}+\omega\times(I\omega) = \tau
\end{equation}
where $\omega$ is the angular velocity vector, $I$ is the inertia matrix, and $\tau$ is a vector of external torques. Assuming that $I = diag(I_{xx}, I_{yy}, I_{zz})$, the body frame rotational equations of motion are given as:
\begin{equation}
\label{rotation}
\left\{
\begin{array}{rl}
I_{xx}\ddot{\phi}=\omega_y \omega_z(I_{yy}-I_{zz})+\tau_x\\
I_{yy}\ddot{\theta}=\omega_x \omega_z(I_{zz}-I_{xx})+\tau_y\\
I_{zz}\ddot{\psi}=\omega_x \omega_y(I_{xx}-I_{yy})+\tau_z
\end{array}
\right.
\end{equation} 
Let $x_1=[x,y,z]^T$, $x_2=[\dot{x},\dot{y},\dot{z}]^T$, $x_3=[\phi,\theta,\psi]^T$, and $x_4=[\omega_x,\omega_y,\omega_z]^T$. Thus, we can write the state space equations for the evolution of our state as:
\begin{equation}
\left\{
\begin{array}{rl}
\dot{x_1} = x_2 \\\\
\dot{x_2} =  \frac{1}{m} \begin{bmatrix} s_{\psi}s_{\theta} k\sum\nolimits \omega_i^2 \\\\
							-c_{\psi}s_{\theta} k\sum\nolimits \omega_i^2 -  \\\\
							c_{\theta} k\sum\nolimits \omega_i^2 -mg 
			\end{bmatrix} - k_dx_2 \\\\
\dot{x_3} = \begin{bmatrix} 1 & \dfrac{s_{\phi}s_{\theta}}{c_{\theta}} & \dfrac{c_{\phi}s_{\theta}}{c_{\theta}} \\\\
0 & c_{\phi} & -s_{\phi}	\\\\
0 & \dfrac{s_{\phi}}{c_{\theta}} & \dfrac{c_{\phi}}{c_{\theta}}
\end{bmatrix} x_4 \\\\
\dot{x_4} =  \begin{bmatrix}
				\dfrac{\tau_{\phi}-(I_{yy}-I_{zz})(\omega_y\omega_z)}{I_{xx}}\\\\
				\dfrac{\tau_{\theta}-(I_{zz}-I_{xx})(\omega_x\omega_z)}{I_{yy}}\\\\
				\dfrac{\tau_{\psi}-(I_{xx}-I_{yy})(\omega_x\omega_y)}{I_{zz}}
				\end{bmatrix}
\end{array}
\right.
\end{equation}
\section{Parameters}
\subsection{Copter Dynamics Parameters}
The necessary parameters for the dynamic modeling of the quadcopter are included in both the linear motion equation and the rotational equation of motion.\\

\subsubsection{Linear Motion}
In the linear motion equations (\ref{thrust}), (\ref{friction}) and (\ref{linear}), the following parameters must be measured directly or indirectly:

\begin{center}
$m$ - mass of the quadcopter\\
$k$ - relation between the RPM of the rotor and the thrust\\
$k_d$ - friction coefficient
\end{center}

\subsubsection{Rotational Motion}
 
In the rotational motion equations, described by (\ref{torque}) and (\ref{rotation}), the parameters to be measured are 
\begin{center}
$I_{xx}$ - moment of inertia in $X$ direction\\
$I_{yy}$ - moment of inertia in $Y$ direction\\
$I_{zz}$ - moment of inertia in $Z$ direction\\
$L$ - Distance between the center of the quadcopter to any of the propellers\\
$b$ - Propeller drag coefficient
\end{center}

\subsection{Motor Dynamics Parameters}
The relation between the angular velocity and input voltage could be linearized as:

\begin{equation}
\dot{\omega}_m=-A\omega_m+Bu+C
\end{equation}
where $u$ is the input voltage. 
Here, the parameters to be determined are $A$, $B$, and $C$.
\end{document}

